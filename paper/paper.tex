% THIS IS SIGPROC-SP.TEX - VERSION 3.1
% WORKS WITH V3.2SP OF ACM_PROC_ARTICLE-SP.CLS
% APRIL 2009
%
% It is an example file showing how to use the 'acm_proc_article-sp.cls' V3.2SP
% LaTeX2e document class file for Conference Proceedings submissions.
% ----------------------------------------------------------------------------------------------------------------
% This .tex file (and associated .cls V3.2SP) *DOES NOT* produce:
%       1) The Permission Statement
%       2) The Conference (location) Info information
%       3) The Copyright Line with ACM data
%       4) Page numbering
% ---------------------------------------------------------------------------------------------------------------
% It is an example which *does* use the .bib file (from which the .bbl file
% is produced).
% REMEMBER HOWEVER: After having produced the .bbl file,
% and prior to final submission,
% you need to 'insert'  your .bbl file into your source .tex file so as to provide
% ONE 'self-contained' source file.
%
% Questions regarding SIGS should be sent to
% Adrienne Griscti ---> griscti@acm.org
%
% Questions/suggestions regarding the guidelines, .tex and .cls files, etc. to
% Gerald Murray ---> murray@hq.acm.org
%
% For tracking purposes - this is V3.1SP - APRIL 2009

\documentclass{dependencies/acm_proc_article-sp}

\usepackage{url}
\usepackage{color}
\usepackage{verbatim}
\usepackage{listings}
\lstset{
  language=C++,             % choose the language of the code
  basicstyle=\small,       % the size of the fonts that are used for the code
  numbers=left,                   % where to put the line-numbers
  numberstyle=\footnotesize,      % the size of the fonts that are used for the line-numbers
  stepnumber=0,                   % the step between two line-numbers. If it is 1 each line will be numbered
  numbersep=5pt,                  % how far the line-numbers are from the code
  backgroundcolor=\color{white},  % choose the background color. You must add \usepackage{color}
  showspaces=false,               % show spaces adding particular underscores
  showstringspaces=false,         % underline spaces within strings
  showtabs=false,                 % show tabs within strings adding particular underscores
  %frame=single,                   % adds a frame around the code
  tabsize=2,              % sets default tabsize to 2 spaces
  captionpos=t,                   % sets the caption-position to bottom
  breaklines=false,        % sets automatic line breaking
  breakatwhitespace=false,    % sets if automatic breaks should only happen at whitespace
  escapeinside={\%}{)}          % if you want to add a comment within your code
}


\begin{document}

\title{ Development Of Parallel Matrix Operations }
%
% You need the command \numberofauthors to handle the 'placement
% and alignment' of the authors beneath the title.
%
% For aesthetic reasons, we recommend 'three authors at a time'
% i.e. three 'name/affiliation blocks' be placed beneath the title.
%
% NOTE: You are NOT restricted in how many 'rows' of
% "name/affiliations" may appear. We just ask that you restrict
% the number of 'columns' to three.
%
% Because of the available 'opening page real-estate'
% we ask you to refrain from putting more than six authors
% (two rows with three columns) beneath the article title.
% More than six makes the first-page appear very cluttered indeed.
%
% Use the \alignauthor commands to handle the names
% and affiliations for an 'aesthetic maximum' of six authors.
% Add names, affiliations, addresses for
% the seventh etc. author(s) as the argument for the
% \additionalauthors command.
% These 'additional authors' will be output/set for you
% without further effort on your part as the last section in
% the body of your article BEFORE References or any Appendices.

\numberofauthors{3} %  in this sample file, there are a *total*
% of EIGHT authors. SIX appear on the 'first-page' (for formatting
% reasons) and the remaining two appear in the \additionalauthors section.
%
\author{
% You can go ahead and credit any number of authors here,
% e.g. one 'row of three' or two rows (consisting of one row of three
% and a second row of one, two or three).
%
% The command \alignauthor (no curly braces needed) should
% precede each author name, affiliation/snail-mail address and
% e-mail address. Additionally, tag each line of
% affiliation/address with \affaddr, and tag the
% e-mail address with \email.
%
% 1st. author
\alignauthor
Nicholas Dexter \\
       \affaddr{RIT}\\
       \email{nxd3734@rit.edu}
% 2nd. author
\alignauthor
Brian Gianforcaro \\
       \affaddr{RIT}\\
       \email{bjg1955@rit.edu}
% 3rd. author
\alignauthor
Jerome Marhic \\
       \affaddr{RIT}\\
       \email{jxm2644@rit.edu}
}
\maketitle
\begin{abstract}
%the formatting guidelines for ACM SIG Proceedings.
%It complements the document \textit{Author's Guide to Preparing
%ACM SIG Proceedings Using \LaTeX$2_\epsilon$\ and Bib\TeX}. This
%source file has been written with the intention of being
%compiled under \LaTeX$2_\epsilon$\ and BibTeX.
%
%The developers have tried to include every imaginable sort
%of ``bells and whistles", such as a subtitle, footnotes on
%title, subtitle and authors, as well as in the text, and
%every optional component (e.g. Acknowledgments, Additional
%Authors, Appendices), not to mention examples of
%equations, theorems, tables and figures.
%
%To make best use of this sample document, run it through \LaTeX\
%and BibTeX, and compare this source code with the printed
%output produced by the dvi file.
\end{abstract}
u
%% A category with the (minimum) three required fields
\category{H.4}{Information Systems Applications}{Miscellaneous}
%%%A category including the fourth, optional field follows...
%
%\terms{Theory}

\keywords{ MongoDB, NoSQL, RIT, Database System Implementation, XPath, GeoSpatial Search, Building Index's in Parallel} % NOT required for Proceedings
%
%\footnotetext[1]{NoSQL databases, also known as structured storage are known for not requiring a fixed table schema and often avoid join operations. The thing that
% links them together is there use of alternate query languages from SQL.}

\section{Introduction}
\section{}
\subsection{}
\subsection{Inserting Records Into The Database}

\subsection{Key Values And Insertion Into Indexes}
\subsection{Page Fetching}
\subsection{Query Processing And Optimizations}
\subsection{Building Indexes In Parallel}
\subsection{XPath Like Queries Of Documents}
\subsection{Expanding Geo-spatial Search}
\subsection{User Authentication}
\newpage

\newpage
%
% The following two commands are all you need in the
% initial runs of your .tex file to
% produce the bibliography for the citations in your paper.
\bibliographystyle{abbrv}
\bibliography{bibliography}  % sigproc.bib is the name of the Bibliography in this case
% You must have a proper ".bib" file
%  and remember to run:
% latex bibtex latex
% to resolve all references
%
% ACM needs 'a single self-contained file'!
%
%APPENDICES are optional
\balancecolumns
% That's all folks!
\end{document}
